\documentclass[letterpaper,notitlepage]{article}
\pagestyle{empty}

\title{Cryptography Final Project Proposal\\Team insert team name\\ HIGHT Algorithm}
\author{Christopher Sasarak\\ Srinivas Sridharan}
\date{\today}

\begin{document}
\maketitle
\thispagestyle{empty}

\section*{Team Information}
\begin{description}
    \item[Algorithm:] HIGHT
    \item[Team Name:] New HIGHT
    \item[Team Members:] Christopher Sasarak, Srinivas Sridharan   
    \item[Email Addresses:] cms5347@rit.edu, sxs9716@rit.edu
\end{description}

\section*{Algorithm}
The HIGHT algorithm\cite{hight} is a lightweight block cipher made for use in simple devices
such as RFID tag. Its simple design lends itself to implementations
not only in software, but also directly in hardware. Indeed, the authors note 
that their hardware implementation of HIGHT can be implemented with only 3048 logic
gates \cite{hight}. Other researchers have also evaluated HIGHT for its viability
in low-power environments and reached the same conclusion 
\cite{hight_implementation}.

HIGHT divides the plain-text into blocks of 64-bits in length and then encrypts
the data using a key of length 128-bits. Before performing the actual encryption, 
HIGHT employs a technique called \emph{key whitening} where it uses whitening
keys generated from the master key to do an initial transformation on the 
plain-text before encrypting it with a 32-round algorithm. At the end of the
encryption, the text output from these 32 rounds are put through an output
transformation using a different set of whitening keys \cite{hight}.

\section*{Attack}
We are planning to implement a partial-key attack against one round of HIGHT.
To do this we will need to find out three pieces of information:the sub-keys that 
are used to encrypt the plain-text and the whitening keys used for both the
input transformation and the output transformation. By testing many different 
known plain-text/cipher-text pairs and exhaustively checking combinations of 
whitening keys we can determine the sub-keys and break the encryption.


\bibliographystyle{plain}
\bibliography{bibliography}
\end{document}
